\documentclass[aspectratio=169,dvipdfmx,14pt,notheorems]{beamer}
%%%% 和文用 %%%%%
\usepackage{bxdpx-beamer}
\usepackage{pxjahyper}
\usepackage{minijs}%和文用
\renewcommand{\kanjifamilydefault}{\gtdefault}%和文用

%%%% スライドの見た目 %%%%%
\usetheme{Madrid}
\usefonttheme{professionalfonts}
\setbeamertemplate{frametitle}[default][center]
\setbeamertemplate{navigation symbols}{}
\setbeamercovered{transparent}%好みに応じてどうぞ)
\setbeamertemplate{blocks}[rounded]
\useinnertheme{circles}
\setbeamertemplate{footline}[page number]
\setbeamerfont{footline}{size=\normalsize,series=\bfseries}
\setbeamercolor{footline}{fg=black,bg=black}
%%%%

%%%% 定義環境 %%%%%
\usepackage{amsmath,amssymb}
\usepackage{amsthm}
\theoremstyle{definition}
\newtheorem{theorem}{定理}
\newtheorem{definition}{定義}
\newtheorem{proposition}{命題}
\newtheorem{lemma}{補題}
\newtheorem{corollary}{系}
\newtheorem{conjecture}{予想}
\newtheorem*{remark}{Remark}
\renewcommand{\proofname}{}
%%%%%%%%%

%%%%% フォント基本設定 %%%%%
\usepackage[T1]{fontenc}%8bit フォント
\usepackage{textcomp}%欧文フォントの追加
\usepackage[utf8]{inputenc}%文字コードをUTF-8
\usepackage[deluxe]{otf}%otfパッケージ
\usepackage{lxfonts}%数式・英文ローマン体を Lxfont にする
\usepackage{bm}%数式太字
%%%%%%%%%%

%%%%% PythonTeX %%%%%
\usepackage[makestderr]{pythontex}
\restartpythontexsession{\thesection}
 
\title{組み込み関数powの知られざる進化}
\subtitle{Unknown Evolution of the Built-in Function pow}
\author[Hayao]{Hayao Suzuki}
\institute[PyCon JP 2021]{PyCon JP 2021}
\date{October 15, 2021}

\begin{document}

\begin{frame}[plain]\frametitle{}
\titlepage %表紙
\end{frame}

\begin{frame}\frametitle{発表に際して}

\begin{block}{GitHubに資料があります}
\begin{itemize}
\item \url{https://github.com/HayaoSuzuki/pyconjp2021}
\end{itemize}
\end{block}

\begin{block}{Twitterのハッシュタグ}
\begin{itemize}
\item \#pyconjp\_1 TBA
\end{itemize}
\end{block}

\begin{block}{PyCon JP Fellow Slack}
\begin{itemize}
\item \#jp-2020-track-1 TBA
\end{itemize}
\end{block}
\end{frame}

\section{Self Introduction}

\begin{frame}\frametitle{Who am I ?}

\begin{block}{お前誰よ}
\begin{description}
\item[Name] Hayao Suzuki(鈴木 駿)
\item[Twitter] \href{https://twitter.com/CardinalXaro}{@CardinalXaro}
\item[Work] Python Programmer
\end{description}
\end{block}

\end{frame}

\begin{frame}\frametitle{Who am I ?}

\begin{block}{Selected Books}
\begin{itemize}
\item \structure{入門 Python 3 第2版}(O'Reilly Japan)
\item \structure{Effective Python 第2版}(O'Reilly Japan)
\item \structure{実践 時系列解析}(O'Reilly Japan) \structure{New!}
\end{itemize}
\end{block}
\url{https://xaro.hatenablog.jp/}にリストがあります。
\end{frame}

\begin{frame}\frametitle{Who am I ?}

\begin{block}{Selected Talks}
\begin{itemize}
\item \structure{レガシーDjangoアプリケーションの現代化}(DjangoCongress JP 2018)
\item \structure{SymPyによる数式処理}(PyCon JP 2018)
\item \structure{Pythonと楽しむ初等整数論}(PyCon mini Hiroshima 2019)
\item \structure{君はcmathを知っているか}(PyCon mini Shizuoka 2020)
\item \structure{インメモリーストリーム活用術}(PyCon JP 2020)
\end{itemize}
\end{block}
\url{https://xaro.hatenablog.jp/}にリストがあります。
\end{frame}

\begin{frame}\frametitle{今日の目標}

\begin{block}{組み込み関数powの知られざる進化}
\begin{itemize}
\item \texttt{pow}関数は遅くともPython 1.4から存在する
\item Pythonに限らず、大抵の言語には\texttt{pow}関数が存在する
\item \texttt{pow}関数は身近な存在
\end{itemize}
\end{block}

\begin{alertblock}{Python 3.8で機能追加}
既に枯れている関数では?
\end{alertblock}
\end{frame}

\section{Conclusion}

\begin{frame}[fragile]\frametitle{Conclusion}
\begin{block}{まとめ}
\begin{itemize}
\item \texttt{pow}関数は身近な存在
\item \texttt{pow}関数は身近な存在
\item \texttt{pow}関数は身近な存在
\item \texttt{pow}関数は身近な存在
\end{itemize}
\end{block}
\texttt{pow}関数はズッ友!
\end{frame}
\end{document}
