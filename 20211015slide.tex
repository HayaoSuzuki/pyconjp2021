\documentclass[aspectratio=169,dvipdfmx,14pt,notheorems]{beamer}
%%%% 和文用 %%%%%
\usepackage{bxdpx-beamer}
\usepackage{pxjahyper}
\usepackage{minijs}%和文用
\renewcommand{\kanjifamilydefault}{\gtdefault}%和文用

%%%% スライドの見た目 %%%%%
\usetheme{Madrid}
\usefonttheme{professionalfonts}
\setbeamertemplate{frametitle}[default][center]
\setbeamertemplate{navigation symbols}{}
\setbeamercovered{transparent}%好みに応じてどうぞ)
\setbeamertemplate{blocks}[rounded]
\useinnertheme{circles}
\setbeamertemplate{footline}[page number]
\setbeamerfont{footline}{size=\normalsize,series=\bfseries}
\setbeamercolor{footline}{fg=black,bg=black}
%%%%

%%%% 定義環境 %%%%%
\usepackage{amsmath,amssymb}
\usepackage{amsthm}
\theoremstyle{definition}
\newtheorem{theorem}{定理}
\newtheorem{definition}{定義}
\newtheorem{proposition}{命題}
\newtheorem{lemma}{補題}
\newtheorem{corollary}{系}
\newtheorem{conjecture}{予想}
\newtheorem*{remark}{Remark}
\renewcommand{\proofname}{}
%%%%%%%%%

%%%%% フォント基本設定 %%%%%
\usepackage[T1]{fontenc}%8bit フォント
\usepackage{textcomp}%欧文フォントの追加
\usepackage[utf8]{inputenc}%文字コードをUTF-8
\usepackage[deluxe]{otf}%otfパッケージ
\usepackage{lxfonts}%数式・英文ローマン体を Lxfont にする
\usepackage{bm}%数式太字
%%%%%%%%%%

%%%%% PythonTeX %%%%%
\usepackage[makestderr]{pythontex}
\restartpythontexsession{\thesection}
 
\title{組み込み関数powの知られざる進化}
\subtitle{Unknown Evolution of the Built-in Function pow}
\author[Hayao]{Hayao Suzuki}
\institute[PyCon JP 2021]{PyCon JP 2021}
\date{October 15, 2021}

\begin{document}

\begin{frame}[plain]\frametitle{}
\titlepage %表紙
\end{frame}

\begin{frame}\frametitle{発表に際して}

\begin{block}{GitHubに資料があります}
\begin{itemize}
\item \url{https://github.com/HayaoSuzuki/pyconjp2021}
\end{itemize}
\end{block}

\begin{block}{Twitterのハッシュタグ}
\begin{itemize}
\item \#pyconjp\_1 TBA
\end{itemize}
\end{block}

\begin{block}{PyCon JP Fellow Slack}
\begin{itemize}
\item \#jp-2021-track-1 TBA
\end{itemize}
\end{block}
\end{frame}

\section{はじめに}

\begin{frame}\frametitle{Who am I ?}

\begin{block}{お前誰よ}
\begin{description}
\item[名前] Hayao Suzuki(鈴木 駿)
\item[Twitter] \href{https://twitter.com/CardinalXaro}{@CardinalXaro}
\item[仕事] Software Developer @ TBA
\end{description}
\end{block}

\end{frame}

\begin{frame}\frametitle{Who am I ?}

\begin{block}{監訳・査読した技術書}
\begin{itemize}
\item \structure{入門 Python 3 第2版}(O'Reilly Japan)
\item \structure{Effective Python 第2版}(O'Reilly Japan)
\item \structure{実践 時系列解析}(O'Reilly Japan) \structure{New!}
\end{itemize}
\end{block}
\url{https://xaro.hatenablog.jp/}にリストがあります。
\end{frame}

\begin{frame}\frametitle{Who am I ?}

\begin{block}{発表リスト}
\begin{itemize}
\item \structure{レガシーDjangoアプリケーションの現代化}(DjangoCongress JP 2018)
\item \structure{SymPyによる数式処理}(PyCon JP 2018)
\item \structure{Pythonと楽しむ初等整数論}(PyCon mini Hiroshima 2019)
\item \structure{君はcmathを知っているか}(PyCon mini Shizuoka 2020)
\item \structure{インメモリーストリーム活用術}(PyCon JP 2020)
\end{itemize}
\end{block}
\url{https://xaro.hatenablog.jp/}にリストがあります。
\end{frame}

\begin{frame}\frametitle{今日の目標}

\begin{block}{組み込み関数\texttt{pow}}
\begin{itemize}
\item \texttt{pow}関数は数のべき乗を返す関数
\item Pythonに限らず、大抵の言語には\texttt{pow}関数が存在する
\end{itemize}
\end{block}

\begin{exampleblock}{Python 3.8で機能追加}
\begin{itemize}
\item 整数$m$を法とする剰余類における逆元が計算できる
\item よくわからない単語を並べるな!
\end{itemize}
\end{exampleblock}
\end{frame}

\begin{frame}\frametitle{今日の目標}
\begin{block}{組み込み関数powの知られざる進化}
\begin{itemize}
\item 「整数$m$を法とする剰余類における逆元」の意味を理解する
\item 「整数$m$を法とする剰余類における逆元」を計算するアルゴリズムを理解する
\end{itemize}
\end{block}
\end{frame}

\section{従来のpow関数}

\begin{frame}\frametitle{今までの\texttt{pow}関数}
\begin{center}
\Large Python 3.7までの\texttt{pow}関数を復習しよう
\end{center}
\end{frame}

\subsection{数のべき乗について}

\begin{frame}\frametitle{整数のべき乗}
\begin{definition}[整数のべき乗]
整数$b$と自然数$n$に対して、\structure{べき乗}$b^{n}$を
\begin{equation*}
b^{n} \triangleq \overbrace{b \times b \times \cdots \times b}^{n\text{個}}
\end{equation*}
と定義する。$b$を\structure{底}、$n$を\structure{指数}と呼ぶ。
\end{definition}

\begin{exampleblock}{整数のべき乗の例}
\begin{equation*}
2^{32} = 4294967296.
\end{equation*}
\end{exampleblock}

\end{frame}

\begin{frame}\frametitle{整数のべき乗}

\begin{block}{Pythonにおけるべき乗}
組み込み関数\texttt{pow}または\texttt{**}演算子を使う。
\end{block}
TODO: pythontexで計算例を載せる
\end{frame}

\subsection{べき乗剰余について}

\begin{frame}\frametitle{べき乗剰余}
\begin{definition}[べき乗剰余]
自然数の底$b$と自然数$n, m$に対して、
\begin{equation*}
b^{n} \bmod{m}
\end{equation*}
を\structure{$m$を法とするべき乗剰余}と定義する。
\end{definition}

\begin{exampleblock}{べき乗剰余の例}
\begin{equation*}
2^{32} \bmod{65535} = 1.
\end{equation*}
\end{exampleblock}

\end{frame}

\begin{frame}\frametitle{べき乗剰余}

\begin{block}{Pythonにおけるべき乗剰余}
\begin{itemize}
\item 組み込み関数\texttt{pow}で効率的に計算できる。
\item \texttt{**}演算子および\texttt{\%}演算子でも計算可能だが非効率。
\end{itemize}
\end{block}
TODO: pythontexで計算例を載せる
\end{frame}

\section{Python 3.8からのpow関数}

\begin{frame}\frametitle{これからの\texttt{pow}関数}
\begin{center}
\Large Python 3.8からの\texttt{pow}関数を理解するために
\end{center}
\end{frame}

\subsection{整数の合同}

\section{まとめ}

\begin{frame}[fragile]\frametitle{Conclusion}
\begin{block}{まとめ}
\begin{itemize}
\item \texttt{pow}関数は身近な存在
\item \texttt{pow}関数は身近な存在
\item \texttt{pow}関数は身近な存在
\item \texttt{pow}関数は身近な存在
\end{itemize}
\end{block}
\texttt{pow}関数はズッ友!
\end{frame}
\end{document}
